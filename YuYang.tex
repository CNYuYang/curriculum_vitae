%Packages
\documentclass[11 pt, a4paper, usenames, dvipsnames]{article}
\usepackage{amsmath,amsfonts, lipsum}
\usepackage[margin = 1in,textwidth = 16cm, includefoot]{geometry}
\usepackage { url, multicol}
\usepackage{sectsty,fontawesome}
\usepackage{enumitem}
\usepackage{parcolumns}
\usepackage{amssymb}
\usepackage{nth}
\usepackage{makecell}
\usepackage{paracol}
 \usepackage{vwcol}
\usepackage{graphicx}
\usepackage[utf8]{inputenc}
\usepackage[x11names]{xcolor}
\usepackage[T1]{fontenc}
\usepackage{ctex}
\usepackage{setspace}       
\usepackage[default]{gillius}
\usepackage[utf8]{inputenc}
\usepackage[T1]{fontenc}
\usepackage[many]{tcolorbox}
\usepackage{hyperref}                                               
\usepackage{color}

%Main color, choose your main color as you wish by changing rgb
\definecolor{CV_Color}{rgb}{0.01, 0.31, 0.59}

%Image path, setup your path accordingly
\graphicspath{ {C:/Users/kerem/Desktop/ResimCV} }


\hypersetup{
    colorlinks=true,
    linkcolor=Black,
    filecolor=magenta,      
    urlcolor=Black,
}

\newtcbox{\mybox}{nobeforeafter,colframe=CV_Color,colback=CV_Color,boxrule=0.5pt,arc=6pt,
boxsep=3pt,left=4pt,right=4pt,top=4pt,bottom=4pt, height=25pt,valign=center} 

\date{}

%Page margins
\geometry{
 a4paper,
 total={170mm,257mm},
 left=8mm,
right = 8mm,
 top=8mm,
bottom = 5mm,
 }
\pagenumbering{gobble}

%Fonts of sections, you can change sizes and color as you wish
\newcommand*{\SubFont}{%
      \fontsize{15}{8}%
\color{Gray}
      \selectfont}

\newcommand*{\RoleFont}{%
      \fontsize{14}{8}%
\color{Black}%
\bf
      \selectfont}

\newcommand*{\DateFont}{%
      \fontsize{10}{8}%
\color{Gray}%
\it
      \selectfont}

\newcommand*{\AchFont}{%
      \fontsize{11}{8}%
\color{Gray}%
      \selectfont}

\newcommand*{\TitleFont}{%
      \fontsize{24}{10}%
\color{CV_Color}
      \selectfont}

\newcommand*{\ProfileFont}{%
      \fontsize{10}{2}%
\color{Black}
      \selectfont}

\newcommand*{\SectionFont}{%
      \fontsize{16}{5}%
\color{CV_Color}%
\bf
\selectfont}

\newcommand{\roundpic}[4][]{
  \tikz\node [circle,draw, minimum width = #2,
    path picture = {
      \node [#1] at (path picture bounding box.center) {
        \includegraphics[width=#3]{#4}};
    }] {};}


\begin{document}
%%%%%%%%%%%%%%%%%%%%%%%%%%%%%%%%%%%%%%%%%%%%%%%%
%%   		INTRODUCTIONPART			  %%		
%%%%%%%%%%%%%%%%%%%%%%%%%%%%%%%%%%%%%%%%%%%%%%%%
\setmainfont{Fira Mono for Powerline}

\setcolumnwidth{0.43\textwidth,0.25\textwidth,0.35\textwidth }
\begin{paracol}{3} 
\begin{flushleft}
% Write name and personal information accordingly
{\textbf{\TitleFont{余阳}}} \vspace{-0.4em}
\end{flushleft}
{\SubFont{Java工程师}}\vspace{0.7em} \\ 
{\ProfileFont 对品质的追求及对雕琢细节的狂热追求,对未知事物有着强烈的好奇心,有着自我驱动的学习力和找到问题答案的能力。}
  \switchcolumn[1]
\begin {center}
\roundpic[xshift=-0.03cm,yshift=0cm]{3.75cm}{3.75cm}{photo}

\end{center}
  \switchcolumn[2]
 \begin{itemize}
% Write contact information.
\item[] \hfill\ProfileFont \href{mailto:744727849@qq.com}{744727849@qq.com} \enskip\color{CV_Color} \verb||\faEnvelopeO 
\item[]\hfill \ProfileFont {+86 17786456292} \enskip\color{CV_Color} \verb||\faPhone
\item[] \hfill \ProfileFont{湖北 武汉} \enskip\color{CV_Color} \verb||\faLocationArrow
\item[] \hfill\ProfileFont\href{https://yuyang.run}{yuyang.run} \enskip \color{CV_Color}\verb||\faLink
\item[]\hfill \ProfileFont\href{https://github.com/CNYuYang} {CNYuYang} \enskip\color{CV_Color} \verb||\faGithub
\end{itemize} 
\end{paracol}

%%%%%%%%%%%%%%%%%%%%%%%%%%%%%%%%%%%%%%%%%%%%%%%%
%%   							 MAIN  PART					 	     	        %%		
%%%%%%%%%%%%%%%%%%%%%%%%%%%%%%%%%%%%%%%%%%%%%%%%
\setcolumnwidth{0.5\textwidth,0.5\textwidth }
\setlength{\columnsep}{2.8em}
\vspace{-2pt}
\begin{paracol}{2} 

%%%%%%%%%%%%%%%%%%%%%%%%%%%%%%%%
%%   			          EXPERIENCE  PART	   		              %%
%%%%%%%%%%%%%%%%%%%%%%%%%%%%%%%%

\section*{\SectionFont\faCubes\enskip  EXPERIENCE}\vspace{-15pt}
\par\noindent\rule{0.22\textwidth}{0.4pt}
%\newline
\begin{itemize}[leftmargin=0pt,align=left,labelwidth=\parindent,labelsep=0pt]

\item[] {\RoleFont Java实习生 } \\
\normalfont 武汉青藤云安全 \\ 
{\DateFont 2020.7 - 至今 \hfill 湖北武汉 } \\ 
{\color{CV_Color}\LARGE - }{蜂巢容器安全开发小组,负责产品配套Jenkins插件开发,以及规则平台新功能的开发。 }


\end{itemize}


%%%%%%%%%%%%%%%%%%%%%%%%%%%%%%%%
%%         PROJECTS  PART	  %%
%%%%%%%%%%%%%%%%%%%%%%%%%%%%%%%%
\section*{\SectionFont\faPaperPlane\enskip PROJECTS}\vspace{-15pt}
\par\noindent\rule{0.18\textwidth}{0.4pt}\vspace{-1pt}
%\newline
\begin{itemize}[leftmargin=0pt,align=left,labelwidth=\parindent,labelsep=0pt]


\item[] \href{https://github.com/CNYuYang/MapReduce}{\textbf{MapReduce分布式计算框架}} \verb||\faGithub\enskip \\ 
使用Java编写的一个分布式计算框架,本项目为MapReduce的最简实现,未引入任何第三方库,Master能保证线程安全的向Worker分配Map、Reduce任务,通过Socket完成Rpc通信。

\item[] \href{https://github.com/CNYuYang/BusTub}{\textbf{BusTub}} \verb||\faGithub \enskip \\ 
正在学习CMU 15-445数据库课程,实现中的一个关系型数据库,目前已完成Buffer Pool部分。在对原C++代码阅读并理解后,该项目使用Java重构,并编写配套测试代码。

\item[] {\textbf{容器安全扫描Jenkins插件}}
\verb||\enskip \\ 
配合容器安全产品的Jenkins插件,在客户CI/CD的过程中,对构建出来的镜像实现自动扫描,分析其风险程度,从而阻断workflow。

\item[] {\textbf{KB机制规则平台}}
\verb||\enskip \\ 
在原有规则平台的基础之上,针对公司新KB机制,编写规则管理页面,提供KB记录的增删改查功能。项目使用easy ui + PHP开发。


\end{itemize}


%%%%%%%%%%%%%%%%%%%%%%%%%%%%%%%%
%%       LANGUAGE  PART       %%
%%%%%%%%%%%%%%%%%%%%%%%%%%%%%%%%
\section*{\SectionFont\faComment\enskip LANGUAGE}\vspace{-15pt}
\par\noindent\rule{0.22\textwidth}{0.4pt}
\vspace{-1em}
\begin{itemize}[leftmargin=0pt,align=left,labelwidth=\parindent,labelsep=0pt]
\setlength{\columnsep}{0.2em}
\begin{multicols}{2}

\item[] {English} \\
{\color{CV_Color} CET-4}

\end{multicols}
\end{itemize}


\switchcolumn[1]


%%%%%%%%%%%%%%%%%%%%%%%%%%%%%%%%
%%   			          EDUCATION  PART	   		              %%
%%%%%%%%%%%%%%%%%%%%%%%%%%%%%%%%
\section*{\SectionFont\faMortarBoard\enskip EDUCATION}\vspace{-15pt}
\par\noindent\rule{0.22\textwidth}{0.4pt} \vspace{0.4pt}
\begin{itemize}[leftmargin=0pt,align=left,labelwidth=\parindent,labelsep=0pt]
	
\item[] {\RoleFont 武汉理工大学} \\
\normalfont 通信工程 \\
{\DateFont 2017.9 - 至今 \hfill 湖北武汉 }\\
{\color{CV_Color}\LARGE - } 获院三好学生,担任电子科技协会会长, GPA 3.439.

\end{itemize}


%%%%%%%%%%%%%%%%%%%%%%%%%%%%%%%%
%% 	   SKILLS  PART 		  %%
%%%%%%%%%%%%%%%%%%%%%%%%%%%%%%%%
\section*{\SectionFont\faStar\enskip SKILLS}\vspace{-15pt}
\par\noindent\rule{0.15\textwidth}{0.4pt} 
\begin{itemize}[leftmargin=0pt,align=left,labelwidth=\parindent,labelsep=0pt]

\item[] {\normalfont {基础}} \\
{\AchFont 数据结构与算法、\href{https://www.bilibili.com/video/av31289365}{\textbf{CMU15-213计算机系统}}、\href{https://www.bilibili.com/video/av70600292}{\textbf{编译原理}}、\href{https://www.bilibili.com/video/av85655193}{\textbf{CMU15-445数据库}}、\href{https://www.bilibili.com/video/av91748150}{\textbf{MIT6.824分布式系统}}}


\item[] {\normalfont {前端}} \\
{\AchFont 熟悉HTML、CSS、JavaScript,掌握Jquery、Vue.js等框架,及Webpack等工具的使用。}

\item[] {\normalfont {后端}} \\
{\AchFont Java基础扎实,对Spring框架比较了解,掌握Spring Boot开发流程
,能搭配MyBatis等框架完成后端开发。熟悉Node.js开发模式,掌握Express等框架开发方式。熟悉PHP开发方式,能独立维护相关工程。能熟练使用Docker容器。}

\item[] {\normalfont {移动端}} \\
{\AchFont 熟悉Android开发,熟练掌握 Android UI 开发,包括各种控件、布局方式、动画和自定义控件的使用方法。}

\end {itemize}

%%%%%%%%%%%%%%%%%%%%%%%%%%%%%%%%
%%       ACHIEVEMENTS  PART	  %%
%%%%%%%%%%%%%%%%%%%%%%%%%%%%%%%%
\section*{\SectionFont\faTrophy\enskip ACHIEVEMENTS}\vspace{-15pt}
\par\noindent\rule{0.26\textwidth}{0.4pt}
\begin{itemize}[leftmargin=0pt,align=left,labelwidth=\parindent,labelsep=0pt]
	
\item[] \href{http://dasai.lanqiao.cn/pages/dasai/index.html}{\normalfont {湖北省 Java A组 三等奖}} \\
{\AchFont 第十届蓝桥杯全国软件和信息技术专业人才大赛省赛}


\item[] \href{http://gjcxcy.bjtu.edu.cn/Index.aspx}{\normalfont {2019年国家级 创业训练类}} \\
{\AchFont 国家级大学生创新创业训练计划}

\item[] \href{https://www.nuedc-training.com.cn/}{\normalfont {湖北省 一等奖 F题 }} \\
{\AchFont 2019年全国大学生电子设计大赛  }

\end {itemize}
\vspace{-0.2em}

\end{paracol}
\end{document}
